% Этот документ когда-то был основан на шаблоне, разработанном в 2014 году
% Данилом Фёдоровых (danil@fedorovykh.ru) 
% для использования в курсе 
% <<Документы и презентации в \LaTeX>>, записанном НИУ ВШЭ
% для Coursera.org: http://coursera.org/course/latex.

\documentclass[a4paper,12pt]{article}

\usepackage{setspace} % Интерлиньяж
\onehalfspacing % Интерлиньяж 1.5
\usepackage{indentfirst}
\usepackage{wasysym}

%%% Работа с русским языком
\usepackage{cmap}					% поиск в PDF
\usepackage{mathtext} 				% русские буквы в фомулах
\usepackage[T2A]{fontenc}			% кодировка
\usepackage[utf8]{inputenc}			% кодировка исходного текста
\usepackage[english,russian]{babel}	% локализация и переносы

%%% Страница
\usepackage{extsizes} % Возможность сделать 14-й шрифт
\usepackage{geometry} % Простой способ задавать поля
	\geometry{top=10mm}
	\geometry{bottom=15mm}
	\geometry{left=20mm}
	\geometry{right=15mm}

%%% Работа с картинками
\usepackage{graphicx}  % Для вставки рисунков
\setlength\fboxsep{3pt} % Отступ рамки \fbox{} от рисунка
\setlength\fboxrule{1pt} % Толщина линий рамки \fbox{}
\usepackage{wrapfig} % Обтекание рисунков текстом

%%% Работа с таблицами
\usepackage{array,tabularx,tabulary,booktabs} % Дополнительная работа с таблицами
\usepackage{longtable}  % Длинные таблицы
\usepackage{multirow} % Слияние строк в таблице

\usepackage{hyperref}
\usepackage[usenames,dvipsnames,svgnames,table,rgb]{xcolor}

\usepackage{csquotes} % Инструменты для ссылок


\usepackage{multicol} % Несколько колонок

\usepackage{hyperref}
\usepackage[usenames,dvipsnames,svgnames,table,rgb]{xcolor}
\hypersetup{				% Гиперссылки
    unicode=true,           % русские буквы в раздела PDF
    pdftitle={Заголовок},   % Заголовок
    pdfauthor={Автор},      % Автор
    pdfsubject={Тема},      % Тема
    pdfcreator={Создатель}, % Создатель
    pdfproducer={Производитель}, % Производитель
    pdfkeywords={keyword1} {key2} {key3}, % Ключевые слова
    colorlinks=true,       	% false: ссылки в рамках; true: цветные ссылки
    linkcolor=red,          % внутренние ссылки
    citecolor=black,        % на библиографию
    filecolor=magenta,      % на файлы
    urlcolor=black           % на URL
}

\usepackage{ulem} %strikethrough


% Работа с кодом
\usepackage{listings}
\lstset{name=bash}

\author{Полина Дроздова}
\title{Основные сведения о дереве кофе}
\date{\today}

\begin{document} % конец преамбулы, начало документа

\maketitle

Кофе (род \textit{Coffea}) — это растение семейства Маревые (Rubiaceae), порядка Горечавкоцветные (Gentianales), клады Asterids. Для производства любимых нами всеми зёрен используют в основном два вида: \textit{C. arabica} и \textit{C. canephora} [\url{http://en.wikipedia.org/wiki/Coffea}]

Недавно (статью в Science вышла 5 сентября 2014 года) был отсеквенирован геном \textit{C.~canephora}. 11 пар хромосом, 710 Mbp. [\url{http://www.sciencemag.org/content/345/6201/1181.full}]. 
Авторы не пишут прямо, но сильно подозреваю, что этот вид предпочли потому, что он диплоид, а \textit{C. arabica} — тетраплоид. Чем больше геном, тем сложнее собирать.


Ещё некоторые полезные ссылки:

\url{https://genomevolution.org/wiki/index.php/File:Totalsequencedtree.png}

\url{http://www.coffeegenome.org/}

\url{http://coffee-genome.org/}

Я утверждаю, что ближайший к кофе секвенированный геном — это либо помидор (\textit{Solanum lycopersicum}), либо картофель (\textit{S. tuberosum}), двух представителей сем. Паслёновые (Solanaceae), входящего в ту же кладу Asterids. Всего десяток лет назад кофе относили к тому же семейству Паслёновые. Проекты по секвенированию генома картофеля и томата живут здесь:

\url{http://www.sgn.cornell.edu/}.

Ещё можно попробовать поработать с генами классического модельного объекта генетики растений \textit{Arabidopsis thaliana}. Насколько я знаю, основной кладезь информации про этот геном здесь:

\url{http://www.arabidopsis.org/}.

Арабидопсис намного дальше от кофе, чем томат и картофель, зато изучен несравнимо лучше.

Я вот нашла статью о том, что у кофе и помидора схожие наборы генов, а на арабидопсисные они как раз не очень похожи:
[\url{http://link.springer.com/article/10.1007\%2Fs00122-005-0112-2}].


А это самая главная страничка, с которой и стоит скачивать данные:

[\textbf{\url{http://coffee-genome.org/download}}]
Паша Добрынин: сказал, что мы можем работать с файлами .fasta с каждой хромосомой. Разбираться со скэффолдами отдельно веселее, но сложнее.

Файл .gff3 в самом начале — это и есть цель нашей работы, поэтому заглядывать в него пока не стоит.

В общем, пока что я предлагаю работать с генами \textit{Arabidopsis thaliana}, \textit{Solanum lycopersicum} и \textit{Solanum tuberosum} и посмотреть, что лучше получится.

\end{document}